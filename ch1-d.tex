%!TEX root = main.tex
\section{바다}

\subsection{자유발화 질문}
\begin{itemize}[noitemsep]
  \item 해산물 좋아하시나요? 어떤 해산물을 좋아하시나요?
  \item 벌교에서는 어떤 해산물을 많이 드시나요? 그 요리의 이름은 무엇인가요?
  \item 낚시나 고기잡이 해보신 적 있으신가요?
  \item 생선 손질해 보신 적 있으신가요? 어떻게 손질하나요?
\end{itemize}

\subsection{목표 어휘}
% \Entry 부분을 수정하시면 됩니다.
% word: 표준어형
% pred: 예상형태
% feat: 유의점
%       feat에 들어가는 인자는 쉼표(,)로 구분되며,
%       ms, sp, dp 세 종류가 있습니다.
%       feat에 들어간 인자가 볼드체로 표시됩니다.
% desc: 설명
% qstn: 유도질문
% advq: 심화질문
%
% 그리고 줄바꿈이 필요할 땐 엔터가 아니라 '\\'를 사용해주세요(중요!!).


\Entry{
  word={바다},
  pred={바다},
  feat={},
  imag={},
  desc={바다와 개펄을 구분하는지 여부},
  qstn={%
    생선이나 해산물은 어디서 나나요? \\
    생선이나 해산물을 잡으려면 어디로 가야 하나요?
  },
  advq={}
}

\Entry{
  word={물고기},
  pred={괴기, 물괴기},
  feat={dp},
  imag={},
  desc={V2 /ㅗ/},
  qstn={생선 같이 물속에서 사는 동물을 뭐라고 하시나요?},
  advq={%
    물고기를 잡았는데 너무 어리면 산 채로 돌려보내잖아요. 이럴 때 물고기를 어떻게 한다고 하시나요? (→ 7.1 살리다) \\
    낚시를 해서 물고기를 잡았어요. 이때 물고기가 어떻게 되었다고 하시나요? (→ 7.11)
  }
}

\newpage
\Entry{
  word={-어(魚)},
  pred={-에},
  feat={dp},
  imag={},
  desc={V1 /ㅓ/},
  qstn={%
    어떤 생선 좋아하세요? \\
    낚시하러 가면 어떤 것을 잡으시나요?
  },
  advq={%
    생선을 오래 보관하려면 냉동실 같은 곳에 차갑게 보관한다고 하잖아요. 이럴 때 생선을 어떻게 한다고 하시나요? (→ 7.5 얼리다) \\
    얼렸던 생선을 매운탕으로 끓여 먹으려면 실내에 좀 두어야 하잖아요. 이럴 때 언 생선을 어떻게 한다고 하시나요? (→ 7.6 녹이다) \\
    생선을 처음 잡으면 펄떡펄떡 뛰어서 손질하기가 어렵잖아요. 이럴 때는 생선을 어떻게 해야 하나요? (→ 7.10 죽이다)
  }
}

\Entry{
  word={어부},
  pred={}, % TODO: 예상형태 기재
  feat={},
  imag={img/refs/어부},
  desc={},
  qstn={생선이나 해산물을 잡아서 파시는 분들을 뭐라고 부르나요?},
  advq={이런 분들이 모여서 사는 곳이 있나요? 어떤 곳인가요?}
}

\Entry{
  word={그물},
  pred={그물},
  feat={},
  imag={img/refs/그물},
  desc={},
  qstn={물고기를 잡는 방법에는 어떤 것이 있을까요?},
  advq={그물은 배 위에서 던지는 방법만 쓰나요? 아니면 물고기나 다른 해산물들을 잡을 수 있는 여러 방법이 있나요?}
}

\newpage
\Entry{
  word={아가미},
  pred={아그미},
  feat={},
  imag={img/refs/아가미},
  desc={},
  qstn={물고기가 물속에서 숨을 쉴 수 있게 해주는 부위를 무엇이라고 하나요?},
  advq={(게, 새우가 이미 관찰되었다면) 물고기 말고 게나 새우에도 아가미와 비슷한 역할을 하는 부위가 있나요? 그 부분을 무엇이라고 부르나요?}
}

\newpage
\Entry{
  word={지느러미},
  pred={쭉지},
  feat={},
  imag={img/refs/지느러미},
  desc={},
  qstn={물고기의 꼬리에서 파닥거려서 헤엄칠 수 있게 하는 부위를 무엇이라고 하나요?},
  advq={꼬리 말고 등이나 옆면에 달린 비슷한 부위도 지느러미라고 하나요?}
}

\Entry{
  word={바닷물},
  pred={겟물, 바닷물},
  feat={},
  imag={img/refs/바닷물},
  desc={},
  qstn={바다는 개울물과 다르게 짠데, 그 짠 물을 무엇이라고 부르나요?},
  advq={짠 물은 모두 그렇게 부르나요? 아니면 바다에 있는 짠물만 그렇게 부르나요?}
}

\Entry{
  word={파랗다},
  pred={필:허다},
  feat={ms,dp},
  imag={img/refs/파랗다},
  desc={\jamoword{p@/ra/h@/da} > 파랗다},
  qstn={날씨가 좋은 날, 바다나 하늘의 색깔이 어떻다고 말씀하시나요?},
  advq={바다의 색깔을 표현하는 다른 말도 있을까요?}
}

\newpage
\Entry{
  word={밀물},
  pred={밀물},
  feat={},
  imag={img/refs/밀물},
  desc={},
  qstn={바다는 물이 들어왔다가, 나갔다가 하잖아요. 이렇게 들어오는 물이나 물이 들어오는 그때를 무엇이라고 부르시나요?},
  advq={날에 따라서 물이 들어오는 정도가 다른가요? 많이 들어오는 날은 무엇이라고 부르나요?}
}

\newpage
\Entry{
  word={썰물},
  pred={}, % TODO: 예상형태 기재
  feat={dp},
  imag={img/refs/썰물},
  desc={/혈물/ > /썰물/},
  qstn={바다는 물이 들어왔다가, 나갔다가 하잖아요. 이렇게 나가는 물이나 물이 빠지는 그때를 무엇이라고 부르시나요?},
  advq={날에 따라서 물이 빠지는 정도가 다른가요? 많이 빠지는 날은 무엇이라고 부르나요?}
}

\Entry{
  word={갯벌},
  pred={껭벌},
  feat={ms},
  imag={img/refs/갯벌},
  desc={개펄과의 의미 차이 여부},
  qstn={%
    조개나 소라를 잡아보신 적이 있으신가요? 어떻게 잡으셨나요? \\
    바다에 물이 싹 빠지고 드러난 질퍽질퍽한 땅을 무엇이라고 부르나요?
  },
  advq={%
    갯벌에서는 어떤 해산물을 잡을 수 있나요? \\
    갯벌 일을 전문적으로 하시는 분들을 부르는 이름이 있나요?
  }
}

% \Entry{
%   word={곶},
%   pred={}, % TODO: 예상형태 기재
%   feat={sp},
%   imag={img/refs/곶},
%   desc={기저 종성 /ㅈ/},
%   qstn={바다 쪽으로, 뾰족하게 뻗은 육지를 무엇이라고 하나요?},
%   advq={}
% }

% \Entry{
%   word={만},
%   pred={}, % TODO: 예상형태 기재
%   feat={},
%   imag={img/refs/만},
%   desc={},
%   qstn={바다 쪽에서 육지 속으로 파고들어 와 있는 곳을 무엇이라고 하나요?},
%   advq={}
% }

\newpage
\Entry{
  word={통발},
  pred={}, % TODO: 예상형태 기재
  feat={},
  imag={img/refs/통발},
  desc={},
  qstn={%
    물고기나 다른 해산물을 잡을 때, 어떤 방법들이 있나요? \\
    10월이 꽃게 철이었는데, 꽃게잡이는 어떻게 하는지 아시나요?
  },
  advq={%
    통발 개수가 너무 적어서 물고기가 잘 안 잡히는 것 같아요. 이럴 때 통발의 개수를 어떻게 해야 좋을까요? (→ 7.2 늘리다) \\
    통발에는 주로 무엇이 잡히나요? \\
    통발 말고 물고기를 가두어서 잡는 방법은 또 없나요?
  }
}

\newpage
\Entry{
  word={새우},
  pred={세비},
  feat={ms,dp},
  imag={img/refs/새우},
  desc={%
    민물 or 바다, 크기에 따른 어휘 차이 \\
    /\jamoword{sa/bqi/}/ > ???
  },
  qstn={얼마 전에 철이었는데, 가재를 닮았고 등이 굽어 있는 해산물을 뭐라고 부르나요? 크기에 따라서 먹는 방법이 조금씩 다릅니다.},
  advq={큰 것과 작은 것을 부르는 이름이 다른가요? 새우는 큰 것만 가리키나요, 작은 것만 가리키나요?}
}

\Entry{
  word={게},
  pred={게:},
  feat={sp},
  imag={img/refs/게},
  desc={V1 /ㅐ-ㅔ/ 대립},
  qstn={집게발을 가졌고, 옆으로 걸어다니는 해산물을 무엇이라고 부르나요?},
  advq={%
    게를 사용한 맛있는 요리를 소개해 주실 수 있나요? \\
    게는 어떻게 손질하나요?
  }
}

\newpage
\Entry{
  word={바위},
  pred={바구, 바우},
  feat={sp, dp},
  imag={img/refs/바위},
  desc={V2 /ㅟ/ \\ 바회… > 바위},
  qstn={%
    꽃게 말고, 갯벌에서 보이는 작은 게들은 보통 어디에 사나요? \\
    고둥이나 따개비는 어디에 붙어서 사나요? \\
    큰 돌을 뭐라고 부르나요?
  },
  advq={바다에 잠겨있어서 보일락말락하는 커다란 돌도 바위라고 부르나요?}
}

\newpage
% new lexicon
\Entry{
  word={널배, 뻘배(비표준어)},
  pred={뻘배},
  feat={},
  imag={},
  desc={},
  qstn={갯벌에서 이동을 편하게 하기 위한 좁고 긴 배 모양의 기구를 뭐라고 부르나요?},
  advq={%
    뻘배는 어떨 때 많이 타나요? \\
    뻘배를 타고 잡는 해산물의 종류는 어떤 것들이 있나요?
  }
}

% new lexicon
\Entry{
  word={해루질(비표준어)},
  pred={해루질},
  feat={},
  imag={},
  desc={},
  qstn={밤에 갯벌이나 얕은 바다에서 손으로 어패류를 잡는 일을 뭐라고 부르나요?},
  advq={해루질로 많이 잡히는 어패류는 어떤 것들이 있나요?}
}